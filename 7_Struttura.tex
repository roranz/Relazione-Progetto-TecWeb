\section{Struttura}
    \subsection{Separazione tra struttura, comportamento e presentazione}
    Uno dei problemi che si è dovuto risolvere per la scrittura del codice HTML è stato quello di evitare l'uso di codice "inline" di altri linguaggi di programmazione all'interno dei file \texttt{.html}, nell'ottica di  garantire il principio di separazione tra struttura, presentazione e comportamento. L'idea è stata quella di frammentare il codice in più parti ricostruendolo successivamente con delle chiamate dinamiche, ogni qualvolta viene caricata una pagina, utilizzando appositi file di codice HTML pure denominati \textit{presets}; alcuni \textit{presets} contengono il mero contenuto da stampare a video, altri invece \textit{template} per costruire la struttura di una pagina web. L'implementazione di questa idea è supportata anche da stringhe racchiuse da una coppia di percentuali, chiamate \textit{placeholders}, necessarie per inserire dinamicamente dei dati generati dal server durante l'invocazione di una pagina (vedi capitolo \blue{\hyperref[Comportamento]{Comportamento}}).\\ I vantaggi che porta questo approccio sono:
    \begin{itemize}
        \item migliore manutenibilità del codice;
        \item codice più pulito.
    \end{itemize}
    
    \subsection{Organizzazione generale del codice}
    Per l'implementazione del codice HTML, si è deciso di seguire le seguenti regole:
    \begin{itemize}
        \item Per evitare troppi tag \texttt{<div>} incapsulati uno dentro l'altro e per creare una struttura più semantica è stato utilizzato il tag \texttt{<section>};
        \item Per incapsulare porzioni di codice che rappresentano aiuti per la navigazione è stato utilizzato il tag \texttt{<nav>};
        \item Le varie opzioni per gli aiuti alla navigazione sono state organizzate con i tag \texttt{<li>} e \texttt{<ul>};
        \item Per rappresentare le intestazioni principali di pagina, sono stati utilizzati i tag \texttt{<h1>} e \texttt{<h2>}; per rappresentare le intestazioni di contenuto si è utilizzato il tag \texttt{<h3>};
        \item Nel testo di contenuto si è utilizzato, dove serviva, il tag \texttt{<time>} per rappresentare date e ore ed il tag \texttt{<abbr>} per rappresentare abbreviazioni;
        \item Per indicare termini in inglese è stato utilizzato l'attributo \texttt{lang=”en”}.
    \end{itemize}
    
    \subsection{Altre regole strutturali}
    Di seguito, sono elencate alcune regole strutturali specifiche per determinate pagine:
    \begin{itemize}
        \item Nella pagina della \textit{Home} è presente una sezione che rappresenta un anteprima dei più recenti articoli presenti nella pagina \textit{News}; tale blocco è stato incapsulato dentro un tag \texttt{<aside>}. Questa decisione è stata presa perché il contenuto all'interno del tag non è indispensabile per la semantica della \textit{Home} e serve solo a supportare la navigazione nel sito;
        \item Nella pagina delle \textit{News}, ma anche nella pagina \textit{Home} tutti gli articoli sono racchiusi dentro un tag \texttt{<article>} per identificare porzioni di testo che sono indipendenti tra di loro;
        \item Per organizzare la struttura di ogni tabella l'intestazione della tabella è stata incapsulata dentro il tag \texttt{<thead>}, mentre il corpo della tabella è stato incapsulato dentro il tag \texttt{<tbody>};
        \item Nella pagina \textit{Registrazione} è stato usato il tag \texttt{<fieldset>} per suddividere la form in due sezioni: la sezione per compilare i dati anagrafici dell'utente e la sezione per compilare i dati di accesso all'area personale del sito;
        \item Nella pagina dedicata all'utente la struttura per visualizzare le informazioni anagrafiche è stata costruita usando elenchi di definizioni, usando i tag \texttt{<dl>}, \texttt{<dt>} e \texttt{<dd>};
        \item Se un utente è autenticato come amministratore nell'area personale è presente un pulsante per passare dalla sezione personale alla sezione che contiene le form per inserire gli articoli e le gare che la società deve pubblicare.
    \end{itemize}