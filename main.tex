\documentclass{article}
    
%Necessario per la traduzione di caratteri in encoding UTF-8 nel linguaggio di LaTeX
    \usepackage[utf8]{inputenc}
    
%Necessario per aggiungere determinati caratteri dell'alfabeto latino e altri simboli come quelli per le frecce
    \usepackage{lmodern,textcomp}

%Necessario per il font utilizzato
    \usepackage{tgheros}
    
%Necessario per la traduzione in italiano
    \usepackage[italian]{babel}

%Necessario per la definizione dei margini e della struttura della pagina
    \usepackage[nomarginpar, left=3cm,right=2cm,top=3cm,bottom=2cm]{geometry}

%Forza posizione immagini
\usepackage{float}

%Necessario per l'utilizzo dello stile di pagina "fancy" che permette l'header e il footer in ogni pagina e mostrare il numero di pagina sul totale
    \usepackage{fancyhdr}
    \usepackage{lastpage}

%Necessari per la creazione dei diagrammi e gli elementi disegnati della pagina
    \usepackage{tikz}
%Aiuta il pacchetto sopra, permettendo l'uso di nodi relativi alla pagina es(\node[anchor=south] at (current page text area.south))
    \usepackage{tikzpagenodes}
    
%XCOLOR
    \usepackage{xcolor}
    \newcommand{\blue}[1]{\textcolor{blue}{#1}}

%Necessario per permettere di utilizzare il parametro [H] per le tabelle e le figure in modo che vengano posizionate esattamente nel punto indicato 
    \usepackage{float}

%Istruzioni per indicare il numero di livelli di gerarchia per i quali generare la numerazione nel documento(Secnumdepth) e nella table of content (tocdepth)
% Valore 5 : Section -> SubSection -> SubSubSection -> Paragraph -> SubParagraph
    \usepackage{titlesec}
    \setcounter{secnumdepth}{5}
    \setcounter{tocdepth}{5}
%Istruzioni per specificare il desiderio di avere il rientro  e il grassetto anche per i paragrafi e sottoparagrafi
    \titleformat{\paragraph}
        {\normalfont\normalsize\bfseries}{\theparagraph}{1em}{}
    \titlespacing*{\paragraph}{0pt}{3.25ex plus 1ex minus .2ex}{1.5ex plus .2ex}
    \titleformat{\subparagraph}
        {\normalfont\normalsize\bfseries}{\thesubparagraph}{1em}{}
    \titlespacing*{\subparagraph}{0pt}{3.25ex plus 1ex minus .2ex}{.75ex plus .1ex}
    
%Necessario per poter includere hyperlink
    \usepackage[hidelinks]{hyperref}
%Tabelle
\usepackage{multirow}


\pagestyle{fancy}
    \fancyhf{}
    \lhead{A.C. Torre Archimede}
    \rhead{Relazione}
    \rfoot{Pagina \thepage / \pageref{LastPage}}
    
    
\title{
    \includegraphics[scale=0.5]{images/logo.png} \\
    \vspace*{1in}
    {
        \Huge \textbf{A.C. Torre Archimede}}\\
        \vspace*{0.25in}
        \textbf{Relazione Tecnologie Web}\\
        \vspace{0.2in}
        \textbf{Versione: 0.1.0}\\
        \vspace{0.1in}
    }

\author{
    \begin{tabular}[t]{c@{}c}
        \hline
        \\
        Michele Bettin - 1216735\\
        Marco Marchiante - 1222397\\
        Stefano Meneguzzo - 1201287\\
        Edoardo Retis - 1100433\\
    \end{tabular}
    \vspace*{0.5in} \\
    Dipartimento di Matematica \\
    \textbf{Università degli Studi di Padova} \\
    \vspace*{0.5in} \\
    \textbf{URL Sito}: \blue{\url{tecweb.studenti.math.unipd.it/mmarchia/}} \\
    \textbf{User Accessibilità}: mbettin (A.C. Torre Archimede)\\
} 
\date{\today}


%--------------------Make usable space all of page
\setlength{\oddsidemargin}{0in} \setlength{\evensidemargin}{0in}
\setlength{\topmargin}{0in}     \setlength{\headsep}{.25in}
\setlength{\textwidth}{6.5in}   \setlength{\textheight}{8.5in}
%--------------------Indention
\setlength{\parindent}{1cm}

\begin{document}
{
    %\fontfamily{qhv}
    \selectfont
    \maketitle
    \newpage
    \tableofcontents
    \newpage

    \section{Abstract}
    Il progetto illustrato da questa relazione ha lo scopo di creare un sito internet che rappresenta una fittizia società calcistica, denominata "Associazione Calcistica Torre Archimede". Tale società ha sede in Italia, più precisamente a Padova e opera sotto la giurisdizione della Federazione Italiana Giuoco Calcio (FIGC). La società ha una squadra di calcio che partecipa a campionati italiani organizzati dalla FIGC, uno stadio, denominato "TorreArchimede Stadium", sede della stessa e che usa per allenare la propria squadra, per disputare gare "in casa", e una tifoseria che ogni settimana si riunisce al botteghino per acquistare un biglietto.\\
    L'obiettivo del sito è di essere un punto di riferimento sia per la società calcistica sia per i fan che navigano su Internet, cercando di raggiungere il maggior numero di persone. Infatti, il sito è usato per rendere partecipi dell'attività della società ai tifosi che già frequentano lo stadio e, al tempo stesso, permettere ai nuovi tifosi di venire a conoscenza del mondo calcistico del Torre Archimede.\\
    Il sito fornisce degli strumenti per permettere a tutte le persone interessate di acquistare online i biglietti di ingresso al campo da gioco, alleggerendo il lavoro di vendita ai tradizionali botteghini.




    \newpage
    \section{Analisi}
    \subsection{Analisi dell'Utenza}
    Il target principale del sito è la tifoseria dell'AC Torre Archimede che, pur essendo abituata al mondo del calcio e familiare con lo stadio della squadra, potrebbe non essere pratica di strumenti informatici. Per questo motivo, il sito internet deve tenerne conto, rendendo il contenuto il più chiaro e semplice possibile.\\
    Un altro punto da considerare sull'utenza del sito è la sua provenienza: la squadra dell'AC Torre Archimede partecipa soltanto a campionati minori, senza mai operare a livello internazionale, presumendo che l'utenza sia italiana, per la maggior parte di loro, residente in zone limitrofe allo stadio. Per questo motivo, la lingua usata nel sito deve essere quella italiana (ad eccezione dei comuni termini di internet, ad esempio: Home, News, Login, ecc.), non richiedendo più versioni alternative in diverse lingue.

    \subsection{Base Informativa}
    Secondo un'analisi preliminare del tipo di utenti, le informazioni che un utente si aspetta di trovare e le domande a cui il sito deve rispondere quando esso vi si accede riguardano l'attività sportiva della società, in particolar modo:
    
    \begin{itemize}
        \item Lo stadio (Dove si trova? Come è fatto?);
        \item La squadra (Chi sono i calciatori? E i tecnici? A quali eventi partecipano?);
        \item Le gare (Quando si giocano? In che stadio? Contro chi? Quanto costa un biglietto?).
    \end{itemize}
    
    \noindent
    Tenendo conto di queste considerazioni, per soddisfare il fabbisogno informativo dell'utente, il sito deve proporre informazioni che riguardano gli argomenti sopra citati, utilizzando come strumenti comunicativi:
    
    \begin{itemize}
        \item Testo descrittivo, se si vogliono utilizzare descrizioni per approfondire determinati argomenti (ad esempio, descrivendo come è fatto lo stadio della squadra);
        \item Articoli, se si vogliono tenere aggiornato l'utente su fatti che accadono nella sfera quotidiana della società;
        \item Elenchi e tabelle, se si vogliono comunicare all'utente una raccolta di dati riferiti ad un certo argomento.
    \end{itemize}


    \subsection{Attori}
    Basandosi sulle necessità del sito, tutti gli utenti si possono raggruppare in tre tipologie: quelli non autenticati, quelli autenticati e gli amministratori.
    
        \subsubsection{Utente non autenticato}
        
        Gli utenti non autenticati sono persone che possono navigare sul sito, ma hanno qualche limitazione. Possono visitare la Homepage, guardare le notizie pubblicate dalla società, leggere la descrizione sullo stadio, consultare l'elenco delle gare in programma, ma non possono acquistare biglietti oppure accedere a sezioni apposite per altri tipi di utenti. Questo tipo di utente può registrarsi al sito per ottenere delle credenziali d'accesso, utili per poter fare l'autenticazione, oppure autenticarsi se ne è già in possesso.
        
        \subsubsection{Utente autenticato}
        
        Gli utenti autenticati sono persone che hanno inserito le proprie credenziali d'accesso e si sono autenticati correttamente. Oltre a poter navigare in tutte le aree del sito, possono acquistare biglietti, scegliendo la gara e il settore dove guardarla, e accedere a una sezione personale dove vengono visualizzati i dati anagrafici dell'utente e l'elenco dei suoi biglietti acquistati.\\ L'utente autenticato può anche effettuare un logout per deautenticarsi.
        
        \subsubsection{Utente amministratore}
        
        Gli amministratori sono utenti autenticati al sito con delle credenziali d'accesso apposite che gli permettono l'ingresso ad una sezione riservata esclusivamente a loro. In quest'area, hanno strumenti per:
        \begin{itemize}
            \item Aggiungere articoli inerenti alla società;
            \item Aggiungere, modificare o rimuovere le partite del calendario.
        \end{itemize}
        Anche loro hanno la possibilità di navigare nel sito e fare acquisti di biglietti.
        

    \subsection{Requisiti funzionali}
    Il primo requisito funzionale che il nostro sito deve adempiere è avere una comunicazione efficace delle informazioni da esporre, cioè organizzare il contenuto in modo tale da poter essere facilmente compreso dalla maggior parte degli utenti che navigano in internet. Per soddisfare questo requisito, ci si è concentrati di più sul contenuto da esporre, sviluppando un design minimale, ma efficace e adattabile a qualsiasi dispositivo.\\
    Un altro requisito da soddisfare è mettere a disposizione dell'utente una piattaforma che gli permetta di fare acquisti online per comprare dei biglietti; a questo scopo, all'interno del sito, è stata implementata una serie di strumenti per:
    
    \begin{itemize}
        \item Controllare le partite in programma e quella passate giocate dalla squadra Torre Archimede;
        \item Acquistare un biglietto;
        \item Controllare i biglietti acquistati;
        \item Cancellare e richiedere eventualmente il rimborso per un biglietto acquistato.
    \end{itemize}
    
    \noindent
    L'ultimo requisito da soddisfare, ma non meno importante, è avere un sito facilmente navigabile e accessibile a tutti. Il sito deve essere accessibile per evitare di limitare la navigazione alle persone con difficoltà.\\
    Per soddisfare questo requisito sono state studiate delle strategie per permettere a qualsiasi categoria di utenza di riuscire a trovare facilmente tutte le informazioni e i contenuti all'interno del sito.\\
    \vspace*{0.05in} \\
    A fine di test, vengono fornite le seguenti credenziali di accesso:
    \vspace*{0.1in} \\
    \begin{center}
        \begin{tabular}{ |p{3cm}|p{3cm}|p{3cm}|p{3cm} } 
        \hline
         & \textbf{Email} & \textbf{Password} \\
        \hline
        \textbf{Utente Autenticato} & utente & utente\\
        \hline
        \textbf{Amministratore} & admin & admin\\
        \hline
        \end{tabular}
    \end{center}
    


    \newpage
    \section{Organizzazione interna}
    \subsection{Divisione dei compiti}
    \subsection{Tempistiche}


    \newpage
    \section{Tecnologie}
    \subsection{Ambiente di sviluppo}
    \subsection{Stack tecnologico}
        \subsubsection{Server}
        \subsubsection{Client}
    \subsection{Configurazione}


    \newpage
    \section{Progettazione}
    \subsection{Struttura delle pagine}
    Per l'organizzazione strutturale delle pagine si è scelto di utilizzare un layout verticale, per rispettare le convenzioni esterne della maggior parte degli utenti e facilitarne la gestione del design, per quando si naviga utilizzando dispositivi diversi dal PC (vedi paragrafo \blue{\hyperref[Presentazione]{Presentazione})}.\\
    Tutte le pagine principali utilizzano il seguente schema:
    \begin{itemize}
        \item \textbf{Header}: è l'intestazione della pagina web dove sono presenti logo e titolo;
        \item \textbf{Menù}: è la parte di pagina contenente una barra di navigazione che indica quali pagine possono essere navigate e comunica se l'utente è autenticato oppure no;
        \item \textbf{Breadcrumb}: è la parte di pagina che comunica all'utente cosa sta visualizzando e gli fornisce un percorso per aiutarlo ad orientarsi;
        \item \textbf{Body}: è la parte di pagina dove sono esposti i contenuti del sito;
        \item \textbf{Footer}: è la parte di pagina dove sono inseriti i contatti della società e i loghi di validazione del sito.
    \end{itemize}
    
    \subsection{Schema e struttura organizzativa}
    Come struttura organizzativa si è deciso di organizzare il sito con una struttura a gerarchia con ampiezza di sette pagine e profondità di massimo due livelli. Questa scelta è stata fatta per migliorare la navigabilità del sito utilizzando una struttura che possa essere facilmente compresa da una persona; per permettergli di sviluppare un chiaro modello mentale del sito, per orientarsi tra le pagine in maniera più naturale e quindi migliore.\\
    Il sito web è suddiviso nelle seguenti pagine:
    \begin{itemize}
        \item \textbf{Home}: è la "vetrina" del nostro sito; dentro questa pagina l'utente può trovare informazioni di vario genere che rappresentano un anteprima dei servizi che il sito offre all'utente;
        \item \textbf{News}: è la pagina dove vengono visualizzate tutte le notizie e gli articoli pubblicati dalla società; per evitare il sovraccarico cognitivo nella pagina viene caricata solo una parte delle notizie pubblicate, ma l'utente le può scorrere tutte usando degli indici predisposti;
        \item \textbf{Stadio}: è la pagina dove è descritta la storia dello stadio dell'AC Torre Archimede;
        \item \textbf{La Squadra}: è la pagina dove viene pubblicato l'elenco dei calciatori e dei dirigenti accompagnatori della società;
        \item \textbf{Calendario}: è la pagina dove sono elencate le ultime gare disputate dalla società;
        \item \textbf{Biglietteria}: è la pagina dove sono elencate le prossime gare in programma che la squadra dovrà disputare; questa pagina, oltre a fornire i dati di tutte le gare, permette di acquistare un biglietto per una gara, tramite la pagina \textbf{Checkout}.
        \item \textbf{Login}: è la pagina dove l'utente può autenticarsi al sito, Se ciò avviene correttamente l'utente viene indirizzato nella propria area personale; altrimenti sono presenti link utili per registrarsi oppure per recuperare la propria password nella pagina \textbf{Recupera password};
        \item \textbf{Area personale/Dashboard admin}: è la pagina dove un utente autenticato può visualizzare e gestire i propri dati anagrafici e l'elenco dei biglietti acquistati; se l'utente autenticato è un amministratore del sito, questa pagina permette l'accesso ad un'ulteriore area riservata, necessaria per gestire diversi aspetti del sito;
        \item \textbf{Registrazione}: è la pagina dove un utente non autenticato potrà inserire i propri dati personali e registrarsi al sito per ottenere le credenziali di accesso necessarie per divenire un utente autenticato e usufruire dei servizi offerti.
    \end{itemize}
    \noindent
    Durante la fase di progettazione si è deciso di organizzare le varie aree del sito per argomenti; si è presa questa decisione per aiutare una persona qualunque a trovare facilmente le informazioni, specialmente quando non sa esattamente cosa sta cercando.
    \newpage
    \section{Base di Dati}
    \subsection{Scelta delle tabelle}
    \subsection{Identificatori primari}
    \subsection{Forma normale}
    \subsection{Utilizzo atteso}

    
    \newpage
    \section{Struttura}
    \subsection{Separazione tra struttura, comportamento e presentazione}
    Uno dei problemi che si è dovuto risolvere per la scrittura del codice HTML è stato quello di evitare l'uso di codice "inline" di altri linguaggi di programmazione all'interno dei file \texttt{.html}, nell'ottica di  garantire il principio di separazione tra struttura, presentazione e comportamento. L'idea è stata quella di frammentare il codice in più parti ricostruendolo successivamente con delle chiamate dinamiche, ogni qualvolta venga caricata una pagina, utilizzando appositi file di codice HTML denominati \textit{presets}; alcuni \textit{presets} contengono il mero contenuto da stampare a video, altri invece, chiamati \textit{template}, forniscono del codice utile a costruire la struttura di una pagina web. L'implementazione di questa idea è supportata anche da stringhe racchiuse da una coppia di percentuali, chiamate \textit{placeholders}, necessarie per inserire dinamicamente dei dati generati dal server durante l'invocazione di una pagina (vedi capitolo \blue{\hyperref[Comportamento]{Comportamento}}).\\ I vantaggi che porta questo approccio sono:
    \begin{itemize}
        \item Migliore manutenibilità del codice;
        \item Codice più pulito.
    \end{itemize}
    
    \subsection{Organizzazione generale del codice}
    Per l'implementazione del codice HTML, si è deciso di seguire le seguenti regole:
    \begin{itemize}
        \item Per creare una struttura più semantica ed evitare tag \texttt{<div>} incorretti, incapsulati uno dentro l'altro, è stato utilizzato il tag \texttt{<section>};
        \item Per incapsulare porzioni di codice che rappresentano aiuti per la navigazione è stato utilizzato il tag \texttt{<nav>};
        \item Le varie opzioni per gli aiuti alla navigazione sono state organizzate con i tag \texttt{<li>} e \texttt{<ul>};
        \item Per rappresentare le intestazioni principali di pagina, sono stati utilizzati i tag \texttt{<h1>} e \texttt{<h2>}; per rappresentare le intestazioni di contenuto si è utilizzato il tag \texttt{<h3>};
        \item Viene utilizzato il tag \texttt{<abbr>} per rappresentare abbreviazioni;
        \item Per indicare termini in inglese è stato utilizzato l'attributo \texttt{lang=”en”}.
    \end{itemize}
    
    \subsection{Altre regole strutturali}
    Di seguito, sono elencate alcune regole strutturali specifiche per determinate pagine:
    \begin{itemize}
        \item Nella pagina della \textit{Home} è presente una sezione che rappresenta un'anteprima dei più recenti articoli presenti nella pagina \textit{News}; tale blocco è stato incapsulato dentro un tag \texttt{<aside>}. Questa decisione è stata presa perché il contenuto all'interno del tag non è indispensabile per la semantica della \textit{Home} e serve solo a supportare la navigazione nel sito;
        \item Nella pagina delle \textit{News}, ma anche nella pagina \textit{Home} tutti gli articoli sono racchiusi dentro un tag \texttt{<article>} per identificare porzioni di testo che sono indipendenti tra di loro;
        \item Per organizzare la struttura di ogni tabella, l'intestazione è stata incapsulata dentro il tag \texttt{<thead>}, mentre il corpo è stato incapsulato dentro il tag \texttt{<tbody>};
        \item Nella pagina \textit{Registrazione} è stato usato il tag \texttt{<fieldset>} per suddividere la form in due sezioni: la sezione per compilare i dati anagrafici dell'utente e la sezione per compilare i dati di accesso all'area personale del sito;
        \item Nella pagina dedicata all'utente, la struttura per visualizzare le informazioni anagrafiche è stata costruita usando elenchi di definizioni, usando i tag \texttt{<dl>}, \texttt{<dt>} e \texttt{<dd>};
        \item Se un utente è autenticato come amministratore nell'area personale, è presente un pulsante per passare alla dashboard di amministrazione, che contiene: le form per inserire gli articoli, un pulsante per inserire una nuova partita e, per ogni partita già presente, la possibilità di modificarla o eliminarla.
    \end{itemize}
    \newpage
    \section{Presentazione}
\label{Presentazione}
    \subsection{Suddivisione dei file}
    Per gestire gli aspetti visivi del sito internet, sono stati prodotti quattro file:
    \begin{itemize}
        \item \texttt{style.css} per gli schermi desktop o con pixel maggiori di 1061px;
        \item \texttt{tablet.css} per i dispositivi tablet o per schermi con pixel minori di 1061px;
        \item \texttt{mobile.css} per dispositivi smartphone o per schermi con pixel minori di 810px;
        \item \texttt{print.css} per stampe con stampanti tradizionali.
    \end{itemize}
    La costruzione di tutti questi file \texttt{.css} hanno lo scopo di ottenere un design capace di adattarsi a qualsiasi dispositivo. A supporto di questa implementazione le misure usate per scrivere le regole CSS sono misure relative come \texttt{em} o percentuali, in modo che l'intero sito possa scalare adeguatamente adattandosi alle scelte dell'utente.
    
    \subsection{CSS Desktop}
    Le caratteristiche del design per gli schermi grandi sono elencati di seguito:
    \begin{itemize}
        \item Il titolo, il logo e la barra di navigazione si espandono in orizzontale, occupando tutto lo schermo; tutti i link della barra di navigazione sono allineati in orizzontale;
        \item Nella \textit{Home} la sezione relativa all'anteprima degli articoli è spostata a destra;
        \item Nella pagina dedicata alle \textit{News} tutti gli articoli sono raggruppati con una grid per evitare troppo scrolling verticale;
        \item L'intestazione e i record delle tabelle si espandono in orizzontale occupando tutto lo schermo a disposizione.
    \end{itemize}
    
    \subsection{CSS Tablet e mobile}
    Le caratteristiche del design per gli schermi di dispositivi portatili sono elencati di seguito:
    \begin{itemize}
        \item Il titolo del sito si toglie lasciando spazio alla visualizzazione soltanto del logo che funge da titolo del sito;
        \item La barra di navigazione viene ridimensionata allineando in verticale i link di navigazione nel sito con i link per l'account dell'utente; se lo schermo si riduce ulteriormente la barra di navigazione si toglie lasciando spazio a due pulsanti che, se premuti, fanno scendere un menù a tendina che mostra i link della barra;
        \item Nella \textit{Home} la sezione relativa all'anteprima degli articoli è spostata al centro seguendo il normale flusso del contenuto;
        \item Nella pagina dedicata alle \textit{News} tutte gli articoli sono visualizzati in verticale uno sotto l'altro;
        \item Il design delle tabelle è stato ridimensionato per trasformarsi in maniera elegante ed evitare lo scrolling orizzontale. Ogni record della tabella viene visualizzato come un blocco dove ogni cella ha la propria intestazione e il proprio dato informativo; le intestazioni incapsulate dentro il tag HTML \texttt{<thead>} si tolgono e ciò impatta sullo scroll verticale aumentando la lunghezza della pagina da scorrere, ma non crea frustrazione perché è accettato dall'utente;
        \item Le form restano invariate.
    \end{itemize}
    
    \subsection{CSS per stampe}
    Le caratteristiche del design per le stampe sono elencati di seguito:
    \begin{itemize}
        \item La barra di navigazione viene nascosta lasciando spazio alla visualizzazione del logo, del titolo e della breadcrumb;
        \item Tutti gli elementi grafici sono stati nascosti lasciando soltanto il mero contenuto con un font con grazie adatto alle stampe;
        \item Le tabelle sono visualizzate normalmente in orizzontale con le celle delimitate da bordi neri e al loro interno il contenuto.
    \end{itemize}
    \newpage
    \section{Comportamento}
\label{Comportamento}
    \subsection{Sviluppo del codice}
    Lo sviluppo dell’aspetto comportamentale del sito è avvenuto in PHP (lato Server) e JavaScript (lato Client).
    Ogni qualvolta il server web registra una richiesta per uno specifico documento, viene eseguito uno script PHP che genera la pagina.
    \subsection{Comportamento lato Server}
    In generale ogni pagina è composta da alcuni elementi fissi come intestazione, barra di navigazione e footer che non cambiano mai e un corpo che differisce da pagina a pagina e molte volte differisce anche all’interno della stessa pagina a seconda della situazione; quindi PHP si occupa di assemblare queste componenti in un documento HTML valido e modificare il corpo della pagina a seconda di ciò che le circostanze richiedono.
    Ad esempio l’area personale dell’utente dovrà mostrare le informazioni relative all’utente attualmente autenticato sulla piattaforma oppure la pagina \textit{Home} dovrà mostrare l’anteprima dei prossime eventi e gli ultimi articoli. Nessuna di queste informazioni e fissa e le pagine vi si devono adattare.
    L’ultimo compito del server è di gestire gli errori come ad esempio tentativi di accesso a risorse ad accesso ristretto (risorse amministrative, file di backend, risorse per utenti autenticati etc…) o problemi a lato server (un malfunzionamento del server web o del database), condizione per cui sono previsti più casi:
    \begin{itemize}
        \item In caso di deliberato accesso (il sito non lo permette quindi tramite URL) a risorse ad accesso limitato o in caso di malfunzionamento del database (fallimento di connessione o errore di query) avviene una ridirezione ad una pagina di errore che fornisce gli strumenti per continuare la navigazione ed un messaggio di errore esplicativo;
        \item In caso di tentato accesso ad una funzionalità non applicabile (come \textit{Login}/\textit{Registrazione} per un utente autenticato o \textit{Logout} per utente ospite) avviene la normale ridirezione all’\textit{Area personale}/\textit{Home} senza che avvenga alcuna comunicazione all’utente.
    \end{itemize}
    L’interattività di una pagina in senso “PHP” è limitata a quelle interazioni che richiedono in qualche modo un cambio o “refresh” della pagina corrente e per qualunque altra interazione è necessario ricorrere a JavaScript poiché operando a livello client è in grado di interagire con il DOM e di conseguenza interagire con la pagina in maniera più "soft" di PHP.
    \subsection{Comportamento lato Client}
    Nel nostro progetto JavaScript ha utilità marginale, nel senso che la maggior parte dei comportamenti interattivi è gestita come descritto precedentemente, ma non per questo è meno importante. Infatti viene utilizzato per modificare dinamicamente il totale durante il processo di checkout e, molto più importante, si occupa di validare l’input nelle form e fornire un feedback all’utente circa l’esito dell’operazione.\\
    Il primo comportamento risiede in un file a se stante e raccogliendo l’evento \texttt{onchange} sul numero di articoli ordinati è in grado di interagire anche con un apposito paragrafo e la lista dei prezzi; in fine i rispettivi dati vengono elaborati e presentati sotto forma di un totale fittizio che viene ricalcolato dal server in fase di ultimazione del checkout.
    Il secondo è più complesso poiché la validazione delle form viene effettuata in più momenti e tutto deve integrarsi con gli errori restituiti dal server aumentando la complessità dell’intero sistema.\\
    La prima validazione viene effettuata “in tempo reale” con lo scopo di informare tempestivamente l’utente di un possibile errore nella compilazione: ogni volta che un campo viene compilato, viene anche validato tramite una specifica funzione JavaScript, attivata dell’evento \texttt{onfocusout}, e si occupa di valutare la correttezza dell’input secondo parametri specifici, generare l’errore e stamparlo a schermo in maniera accessibile per le tecnologie assistive e correttamente presentato per essere facilmente identificato nella pagina.\\
    La seconda validazione viene effettuata prima che la form venga inviata come controllo di sanità generale poiché è possibile che l’utente non abbia compilato tutti i campi e rimbalzare il controlllo al server è poco usabile per l’utente: quando una form viene inviata al server, l’evento \texttt{onsubmit} della form aziona una particolare funzione che effettua i controlli per tutta la form e consente l’invio solo se nessun errore si è verificato. Per evitare la duplicazione del codice questa classe di funzioni  di validazione si limita ad eseguire le stesse funzioni del punto precedente, che aggiornano a loro volta una variabile per registrare la presenza di errori nella compilazione della form. Questa variabile viene poi letta da questa funzione (e salvata), resettata e in base al suo valore la form verrà inviata o meno.
    \newpage
    \section{Accessibilità}
    \subsection{Strategie d'implementazione}
    Si è cercato attentamente di rendere il sito accessibile per permetterne l'utilizzo a qualsiasi fascia di utenza, ponendo il focus del lavoro svolto su diversi problemi:
    \begin{itemize}
        \item Permettere agli utenti diversamente abili che utilizzano gli screen reader di comprendere meglio il contenuto del sito;
        \item Permettere agli utenti ipovedenti e daltonici una navigazione senza problemi, utilizzando font, contrasti e colori in maniera accessibile.
        \item Permettere agli utenti che utilizzano dispositivi mobile di navigare facilmente nel sito.
    \end{itemize}
    Per gestire questi aspetti sono stati individuati i seguenti punti critici: immagini, aiuti alla navigazione, testo, tabelle, form, navigazione con dispositivi mobile, messaggi in Javascript.\\
    Per ogni punto critico sono stati studiati ed implementati degli strumenti per l'accessibilità che poi sono stati successivamente verificati con appositi test.
    \subsubsection{Immagini}
    Si è cercato di rendere accessibili le immagini tramite appositi attributi \texttt{alt}, i quali permettono agli screen readers di dare un'appropriata descrizione di ciò che l'immagine rappresenta,in modo da non tagliare fuori gli utenti con disabilità visive.
    \subsubsection{Aiuti alla navigazione}
    Il sito è stato dotato di aiuti per la navigazione come il pulsante per saltare la barra di navigazione e andare direttamente al contenuto, utile per evitare che lo screen reader debba ogni volta rileggere il menù all’apertura di ogni pagina, o la freccia per tornare su una volta arrivati a fondo pagina. Chiaramente tutti questi aiuti sono raggiungibili tramite tabulazione.
    \subsubsection{Testo}
    Nella scelta del testo si è optato per un font senza grazie nella versione del sito dedicata agli schermi. Si è scelta una dimensione adeguata che permettesse alla maggior parte degli utenti una lettura senza affaticare la vista o di creare problemi alle persone con disabilità cognitive. I testi si è cercato di comporli in modo che fossero di facile comprensione. Si è evitato il corsivo, i testi lampeggianti o sottolineature inutili che andrebbero a creare ambiguità con i link.\\
    I link esterni sono presenti solo nella biglietteria e il reindirizzamento esterno viene segnalato a parole in maniera esplicita.
    \subsubsection{Tabelle}
    Per gli utenti con disabilità di tipo visivo risulta abbastanza complicato trasformare un oggetto bidimensionale come la tabella in un qualcosa monodimensionale come può essere l’audio di uscita di uno screen reader. Per questo motivo, si è deciso di rendere le tabelle accessibili inserendo un tag \texttt{<caption>} subito dopo il tag \texttt{<table>} per fornire una breve descrizione di quello a cui la tabella fa riferimento e l'attributo \texttt{scope="col"} nelle celle delle intestazioni per supportare gli screen readers alla lettura delle tabelle. Sfortunatamente Le tabelle in questo sito non possiedono un attributo che possa da se fare da chiave per l'intera riga e di conseguenza l'inserimento dell'attributo \texttt{scope="row"} è stato evitato poiché avrebbe causato solo confusione.
    \subsubsection{Form}
    Riguardo alle form, si è utilizzato il tag \texttt{<label>} per poter adeguatamente identificare ogni relativo campo input. Nel caso di errori si è deciso di inserire dei messaggi (PHP o Javascript) sotto i relativi campi input che avranno il bordo color verde o rosso a seconda che siano stati o meno compilati correttamente. I messaggi inseriti sono presentati in modo che siano facilmente localizzabili da tutte le categorie di utenti. In particolare sono stati pensati per essere compresi dalle tecnologie assistive tramite il collegamento al controllo ha ha generato l'errore e l'impiego dell'attributo \texttt{aria-live="assertive"} per fari si che vengano presentati all'utente immediatamente. Inoltre quando un errore viene rilevato durante il tentativo di inviare il form al server, il focus viene spostato sul primo elemento in ordine naturale ad avere un errore.
    \subsubsection{Navigazione con dispositivi mobile}
    Durante la fase di costruzione del progetto, si è cercando di fare in modo che il design del sito possa mantenersi anche con differenti dispositivi hardware, proprio perché il mondo del mobile presenta dimensioni talmente variegate da non permettere a chi progetta siti web di potersi concentrare su 3/4 dimensioni di riferimento. Per questo motivo si è deciso di seguire la politica del responsive design inserendo adeguatamente dei breakpoints per adattare le pagine web in modo da renderle accessibili su tutti i dispositivi. Inoltre, si è deciso che per quanto riguarda gli smartphone fosse necessario mettere a disposizione dell'utente una così detta "icona ad hamburger" per permettere una presentazione più elegante senza perdere in praticità, che di fatto va a nascondere la barra di navigazione dividendola tra parte di contenuti e parte dedicata alla gestione dell'area personale.  Per quanto riguarda l'intestazione del sito, si è deciso di lasciare il logo rimuovendo il titolo per motivi di praticità ed eleganza. Infine, si è cercato di fare in modo che le tabelle si trasformassero in modo elegante nel passaggio da schermi di dimensioni standard a schermi piccoli utilizzando l’attributo \texttt{data-label} per accostare la relativa intestazione al contenuto della singola cella.
    \subsubsection{Messaggi in Javascript}
    Una particolare attenzione è stata rivolta alla gestione degli errori via Javascript, in modo che non fosse possibile una ridondanza tra errori Javascript ed errori PHP, per evitare un sovraccarico cognitivo che andrebbe ad impedire una corretta fruizione del sito da parte dell’utente. Il codice Javascript va quindi a controllare che non siano già presenti messaggi di errore in PHP e nel caso andrà a sostituirli.
    \subsubsection{Colori}
    Sono stati scelti colori specifici che rispecchino i requisiti di accessibilità e non creino problemi a utenti affetti da qualsiasi tipo di daltonismo. Di seguito, viene caricato un estratto della palette scelta:
    \begin{center}
        \includegraphics[scale=0.8]{images/palette.png}
    \end{center}
    
    \newpage
    \subsection{Test}
    \label{Test}
    Per verificare che il sito funzionasse correttamente dal punto di vista dell'accessibilità, sono stati fatti i seguenti test:
    \begin{itemize}
        \item Si è usato W3C Validator per la validazione di tutte le pagine HTML;
        \item Tramite lo screen reader NVDA si è controllata la corretta accessibilità al sito per persone con disabilità visive;
        \item Si è verificato che ci sia un corretto contrasto a livello di scelta dei colori utilizzati in ciascuna delle singole pagine;
        \item Si è  verificato che il sito venisse visualizzato correttamente anche con differenti browser;
        \item Si è svolto il questionario di accessibilità del dipartimento di matematica, effettuando tutti i test per una durata complessiva di circa 6 ore complessive.
    \end{itemize}
    \newpage
    \section{Usabilità}
    \subsection{Area sicura}
    Per migliorare l'usabilità del sito si è studiata la struttura delle pagine web in modo tale da visualizzare nell'area così detta "above the fold" di ogni dispositivo le seguenti informazioni:
    \begin{itemize}
        \item Il nome del sito scritto nell'intestazione che risponde alla domanda "Dove sono?" (è stato usato l'acronimo "AC" perché si presume che l'utente conosca il suo significato);
        \item Il nome delle pagine elencate nella barra di navigazione che risponde alla domanda "Dove posso andare?";
        \item Il titolo e le informazioni essenziali del contenuto della pagina che risponde alla domanda "Di che cosa si tratta?".
    \end{itemize}
    
    \subsection{Altri strumenti per l'usabilità}
    Sempre nell'ottica di migliorare l'usabilità del sito ed evitare disorientamento da parte dell'utente sono stati implementati i seguenti aiuti alla navigazione:
    \begin{itemize}
        \item Rispettando le convenzioni esterne i link non visitati sono colorati di blu, mentre quelli visitati sono colorati di viola; insieme alla breadcrumb questi aiuti permettono all'utente di capire che pagina ha visitato e quale no rispondendo alle domande "Dove posso andare?" e "Come sono arrivato qui?";
        \item A piè di pagina sono state aggiunte informazioni su chi gestisce il sito che risponde alla domanda "Da chi è gestita questa pagina?";
        \item In basso a destra è stata aggiunta una freccia che, se premuta, porta direttamente all'inizio del contenuto della pagina;
        \item Nella pagina \textit{News} è stata aggiunta una casella di ricerca per permettere all'utente di fare interrogazioni più precise nel sito per trovare articoli specifici rispondendo alla domanda "Dove posso trovare informazioni più approfondite?";
        \item Nella pagina \textit{Biglietteria} sono stati aggiunti link che portano a siti esterni per permettere all'utente di capire dove acquistare i biglietti se è impossibilitato ad acquistarli dentro il sito;
        \item Nei dispositivi desktop la barra di navigazione non ha menù a tendina; nei dispositivi mobile la barra di navigazione ha due pulsanti che, se premuti, mostrano due menù a tendina che non generano frustrazione;
        \item Con un messaggio in verde, si comunica se l'utente si è autenticato o ha completato qualche operazione con successo.
    \end{itemize}


    \newpage
    \section{Conclusioni}
Il progetto è stato portato a termine entro i tempi previsti senza incontrare particolari problemi ed ogni membro del gruppo ha saputo gestire bene le tecnologie utilizzate per creare un sito internet efficiente dal punto di vista dei contenuti, dell'accessibilità e dell'usabilità. Sebbene sia stato usato HTML5 e CSS3 alcune funzioni di questi linguaggi, come ad esempio le canvas, non sono state utilizzate. Questa scelta è stata fatta perché l'utenza e le richieste del sito necessitano di poche funzionalità.\\
Infine sono stati eseguiti sul sito tutti i test di accessibilità riportati all'indirizzo \blue{\url{https://web.math.unipd.it/accessibility-dev/}}.

\begin{center}
    \includegraphics[scale=0.5]{images/test.png}
\end{center}
    \newpage
}

\end{document}