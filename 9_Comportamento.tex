\section{Comportamento}
\label{Comportamento}
    \subsection{Sviluppo del codice}
    Lo sviluppo dell’aspetto comportamentale del sito è avvenuto in PHP (lato Server) e JavaScript (lato Client).
    Ogni qualvolta il server web registra una richiesta per uno specifico documento, viene eseguito uno script PHP che genera la pagina.
    \subsection{Comportamento lato Server}
    In generale ogni pagina è composta da alcuni elementi fissi come intestazione, barra di navigazione e footer che non cambiano mai e un corpo che differisce da pagina a pagina e molte volte differisce anche all’interno della stessa pagina a seconda della situazione; quindi PHP si occupa di assemblare queste componenti in un documento HTML valido e modificare il corpo della pagina a seconda di ciò che le circostanze richiedono.
    Ad esempio l’area personale dell’utente dovrà mostrare le informazioni relative all’utente attualmente autenticato sulla piattaforma oppure la pagina \textit{Home} dovrà mostrare l’anteprima dei prossime eventi e gli ultimi articoli. Nessuna di queste informazioni e fissa e le pagine vi si devono adattare.
    L’ultimo compito del server è di gestire gli errori come ad esempio tentativi di accesso a risorse ad accesso ristretto (risorse amministrative, file di backend, risorse per utenti autenticati etc…) o problemi a lato server (un malfunzionamento del server web o del database), condizione per cui sono previsti più casi:
    \begin{itemize}
        \item In caso di deliberato accesso (il sito non lo permette quindi tramite URL) a risorse ad accesso limitato o in caso di malfunzionamento del database (fallimento di connessione o errore di query) avviene una ridirezione ad una pagina di errore che fornisce gli strumenti per continuare la navigazione ed un messaggio di errore esplicativo;
        \item In caso di tentato accesso ad una funzionalità non applicabile (come \textit{Login}/\textit{Registrazione} per un utente autenticato o \textit{Logout} per utente ospite) avviene la normale ridirezione all’\textit{Area personale}/\textit{Home} senza che avvenga alcuna comunicazione all’utente.
    \end{itemize}
    L’interattività di una pagina in senso “PHP” è limitata a quelle interazioni che richiedono in qualche modo un cambio o “refresh” della pagina corrente e per qualunque altra interazione è necessario ricorrere a JavaScript poiché operando a livello client è in grado di interagire con il DOM e di conseguenza interagire con la pagina in maniera più "soft" di PHP.
    \subsection{Comportamento lato Client}
    Nel nostro progetto JavaScript ha utilità marginale, nel senso che la maggior parte dei comportamenti interattivi è gestita come descritto precedentemente, ma non per questo è meno importante. Infatti viene utilizzato per modificare dinamicamente il totale durante il processo di checkout e, molto più importante, si occupa di validare l’input nelle form e fornire un feedback all’utente circa l’esito dell’operazione.\\
    Il primo comportamento risiede in un file a se stante e raccogliendo l’evento \texttt{onchange} sul numero di articoli ordinati è in grado di interagire anche con un apposito paragrafo e la lista dei prezzi; in fine i rispettivi dati vengono elaborati e presentati sotto forma di un totale fittizio che viene ricalcolato dal server in fase di ultimazione del checkout.
    Il secondo è più complesso poiché la validazione delle form viene effettuata in più momenti e tutto deve integrarsi con gli errori restituiti dal server aumentando la complessità dell’intero sistema.\\
    La prima validazione viene effettuata “in tempo reale” con lo scopo di informare tempestivamente l’utente di un possibile errore nella compilazione: ogni volta che un campo viene compilato, viene anche validato tramite una specifica funzione JavaScript, attivata dell’evento \texttt{onfocusout}, e si occupa di valutare la correttezza dell’input secondo parametri specifici, generare l’errore e stamparlo a schermo in maniera accessibile per le tecnologie assistive e correttamente presentato per essere facilmente identificato nella pagina.\\
    La seconda validazione viene effettuata prima che la form venga inviata come controllo di sanità generale poiché è possibile che l’utente non abbia compilato tutti i campi e rimbalzare il controlllo al server è poco usabile per l’utente: quando una form viene inviata al server, l’evento \texttt{onsubmit} della form aziona una particolare funzione che effettua i controlli per tutta la form e consente l’invio solo se nessun errore si è verificato. Per evitare la duplicazione del codice questa classe di funzioni  di validazione si limita ad eseguire le stesse funzioni del punto precedente, che aggiornano a loro volta una variabile per registrare la presenza di errori nella compilazione della form. Questa variabile viene poi letta da questa funzione (e salvata), resettata e in base al suo valore la form verrà inviata o meno.