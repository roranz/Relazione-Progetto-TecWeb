\section{Usabilità}
    \subsection{Area sicura}
    Per migliorare l'usabilità del sito si è studiata la struttura delle pagine web in modo tale da visualizzare nell'area così detta "above the fold" di ogni dispositivo le seguenti informazioni:
    \begin{itemize}
        \item Il nome del sito scritto nell'intestazione che risponde alla domanda "Dove sono?" (è stato usato l'acronimo "AC" perché si presume che l'utente conosca il suo significato);
        \item Il nome delle pagine elencate nella barra di navigazione che risponde alla domanda "Dove posso andare?";
        \item Il titolo e le informazioni essenziali del contenuto della pagina che risponde alla domanda "Di che cosa si tratta?".
    \end{itemize}
    
    \subsection{Altri strumenti per l'usabilità}
    Sempre nell'ottica di migliorare l'usabilità del sito ed evitare disorientamento da parte dell'utente sono stati implementati i seguenti aiuti alla navigazione:
    \begin{itemize}
        \item Rispettando le convenzioni esterne i link non visitati sono colorati di blu, mentre quelli visitati sono colorati di viola; insieme alla breadcrumb questi aiuti permettono all'utente di capire che pagina ha visitato e quale no rispondendo alle domande "Dove posso andare?" e "Come sono arrivato qui?";
        \item A piè di pagina sono state aggiunte informazioni su chi gestisce il sito che risponde alla domanda "Da chi è gestita questa pagina?";
        \item In basso a destra è stata aggiunta una freccia che, se premuta, porta direttamente all'inizio del contenuto della pagina;
        \item Nella pagina \textit{News} è stata aggiunta una casella di ricerca per permettere all'utente di fare interrogazioni più precise nel sito per trovare articoli specifici rispondendo alla domanda "Dove posso trovare informazioni più approfondite?";
        \item Nella pagina \textit{Biglietteria} sono stati aggiunti link che portano a siti esterni per permettere all'utente di capire dove acquistare i biglietti se è impossibilitato ad acquistarli dentro il sito;
        \item Nei dispositivi desktop la barra di navigazione non ha menù a tendina; nei dispositivi mobile la barra di navigazione ha due pulsanti che, se premuti, mostrano due menù a tendina che non generano frustrazione;
        \item Con un messaggio in verde, si comunica se l'utente si è autenticato o ha completato qualche operazione con successo.
    \end{itemize}

