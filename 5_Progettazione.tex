\section{Progettazione}
    \subsection{Struttura delle pagine}
    Per l'organizzazione strutturale delle pagine si è scelto di utilizzare un layout verticale, per rispettare le convenzioni esterne della maggior parte degli utenti e facilitare la gestione del design, per quando si naviga utilizzando dispositivi diversi dal PC (vedi capitolo \blue{\hyperref[Presentazione]{Presentazione}}).\\
    Tutte le pagine principali utilizzano il seguente schema:
    \begin{itemize}
        \item \textbf{Header}: è l'intestazione della pagina web dove sono presenti logo e titolo;
        \item \textbf{Menù}: è la parte di pagina contenente una barra di navigazione che indica le sezioni in cui è diviso il sito e comunica all'utente se è autenticato oppure no;
        \item \textbf{Breadcrumb}: è la parte di pagina che comunica all'utente cosa sta visualizzando e e lo aiuta ad orientarsi fornendogli il percorso per arrivarci;
        \item \textbf{Body}: è la parte di pagina dove sono esposti i contenuti del sito;
        \item \textbf{Footer}: è la parte di pagina dove sono inseriti i contatti della società, gli autori del sito e i badge di validazione.
    \end{itemize}
    
    \subsection{Schema e struttura organizzativa}
    Come struttura organizzativa si è deciso di organizzare il sito con una struttura a gerarchia con ampiezza di otto pagine e profondità di massimo tre livelli. Questa scelta è stata fatta per migliorare la navigabilità del sito utilizzando una struttura che possa essere facilmente compresa da una persona comune; per permettergli di sviluppare un chiaro modello mentale del sito, per orientarsi tra le pagine in maniera più naturale e quindi migliore.\\
    Il sito web è suddiviso nelle seguenti pagine:
    \begin{itemize}
        \item \textbf{Home}: è la "vetrina" del nostro sito; dentro questa pagina l'utente può trovare informazioni di vario genere che rappresentano un'anteprima dei servizi offerti all'utente;
        \item \textbf{News}: è la pagina dove vengono visualizzate tutte le notizie e gli articoli pubblicati dalla società; per evitare il sovraccarico cognitivo nella pagina viene caricata solo una parte delle notizie pubblicate, ma l'utente le può scorrere tutte usando degli indici predisposti e una casella di ricerca per eventuali ricerche più specifiche;
        \item \textbf{Stadio}: è la pagina dove è descritta la storia dello stadio dell'AC Torre Archimede;
        \item \textbf{La Squadra}: è la pagina dove viene pubblicato l'elenco dei calciatori e dei tecnici della società;
        \item \textbf{Calendario}: è la pagina dove sono elencate le ultime gare disputate dalla A.C. Torre Archimede;
        \item \textbf{Biglietteria}: è la pagina dove sono elencate le prossime gare del campionato in programma; questa pagina, oltre a fornire i dati di tutte le gare, permette di acquistare biglietti per quelle che si disputano nel Torre Archimede Stadium, fornendo un sistema di \textbf{Checkout}. Per le altre, viene fornito un collegamento esterno verso il sito dello stadio nel quale si disputa la partita;
        \item \textbf{Login}: è la pagina dove l'utente può autenticarsi al sito, se ciò avviene correttamente l'utente viene indirizzato nella propria area personale; altrimenti sono presenti link utili per registrarsi oppure per recuperare la propria password nella pagina \textbf{Recupera password};
        \item \textbf{Area personale}: è la pagina dove un utente autenticato può visualizzare e gestire i propri dati anagrafici e l'elenco dei biglietti acquistati; se l'utente autenticato è un amministratore del sito, questa pagina permette l'accesso ad un'ulteriore pagina (\textbf{Dashboard admin}), necessaria per amministrare diversi aspetti del sito;
        \item \textbf{Registrazione}: è la pagina dove un utente non autenticato potrà inserire i propri dati personali e registrarsi al sito per ottenere le credenziali di accesso necessarie per divenire un utente autenticato e usufruire dei servizi offerti.
    \end{itemize}
    \noindent
    Durante la fase di progettazione si è deciso di organizzare le varie aree del sito per argomenti; si è presa questa decisione per aiutare una persona qualunque a trovare facilmente le informazioni, specialmente quando non sa esattamente cosa sta cercando.